\documentclass[french]{article}
\usepackage[T1]{fontenc}
\usepackage[utf8]{inputenc}
\usepackage[french]{babel}
\usepackage{amsmath}
\usepackage{forest}

\begin{document}

\subsection*{Dérivation TAG : « Pierre casse un verre de vin »}

\subsubsection*{1. Arbres élémentaires}

\textbf{1.1. Arbres initiaux}
\\
\\
Soit l’arbre $\alpha_{\text{casse}}$ (structure transitive) :
\par\noindent
\begin{forest}
for tree={align=center,l sep=8mm,s sep=6mm,scale=0.92}
[S
  [GN$_{\downarrow}$]
  [GV
    [V [casse]]
    [GN$_{\downarrow}$]
  ]
]
\end{forest}

\medskip
Soit l’arbre $\alpha_{\text{Pierre}}$ :
\par\noindent
\begin{forest}
for tree={align=center,l sep=8mm,s sep=6mm,scale=0.92}
[GN [Pierre]]
\end{forest}

\medskip
Soit l’arbre $\alpha_{\text{verre}}$ :
\par\noindent
\begin{forest}
for tree={align=center,l sep=8mm,s sep=6mm,scale=0.92}
[GN [N [verre]]]
\end{forest}

\medskip
Soit l’arbre $\alpha_{\text{vin}}$ :
\par\noindent
\begin{forest}
for tree={align=center,l sep=8mm,s sep=6mm,scale=0.92}
[GN [N [vin]]]
\end{forest}
\clearpage
\textbf{1.2. Arbres auxiliaires}

Soit l’arbre auxiliaire $\beta_{\text{un}}$ :
\par\noindent
\begin{forest}
for tree={align=center,l sep=8mm,s sep=6mm,scale=0.92}
[GN
  [Det [un]]
  [GN$^\ast$]
]
\end{forest}

\medskip
Soit l’arbre auxiliaire $\beta_{\text{de}}$ (modifieur optionnel du nom) :
\par\noindent
\begin{forest}
for tree={align=center,l sep=8mm,s sep=6mm,scale=0.92}
[GN
  [GN$^\ast$]
  [GP
    [P [de]]
    [GN$_{\downarrow}$]
  ]
]
\end{forest}

\subsection*{Dérivation}

\begin{enumerate}
  \item Partir de \textbf{$\alpha_{\text{casse}}$}.
  \item Substituer \textbf{$\alpha_{\text{Pierre}}$} sur le nœud \(\mathrm{GN}_{\downarrow 1}\) (sujet).
  \item Substituer \textbf{$\alpha_{\text{verre}}$} sur le nœud \(\mathrm{GN}_{\downarrow 2}\) (objet).
  \item Adjoindre \textbf{$\beta_{\text{un}}$} au \(\mathrm{GN}\) objet.
  \item Adjoindre \textbf{$\beta_{\text{de}}$} au \(\mathrm{GN}\) interne (\emph{verre}) — modifieur optionnel.
  \item Substituer \textbf{$\alpha_{\text{vin}}$} sur le nœud \(\mathrm{GN}_{\downarrow 3}\) (complément du nom).
\end{enumerate}

\subsection*{Dérivation TAG de « Pierre boit un verre de vin »}

On construit \(\alpha_{\text{boit}}\) ayant la même structure transitive :
\[
\begin{forest}
for tree={align=center,l sep=8mm,s sep=6mm,scale=0.92}
[S
  [GN$_{\downarrow}$]
  [GV
    [V [boit]]
    [GN$_{\downarrow}$]
  ]
]
\end{forest}
\]

\medskip
Arbre \(\alpha_{\text{un-verre-de}}\) :
\[
\begin{forest}
for tree={align=center,l sep=10mm,s sep=8mm, scale=0.92}
[GN
  [Det [un]]
  [N'
    [N [verre]]
    [GP
      [P [de]]
      [N]
    ]
  ]
]
\end{forest}
\]


\paragraph{Justification.}
Si l’on sépare \og un\fg{}, \og verre\fg{}, \og de\fg{}, \og vin\fg{} en arbres élémentaires, c’est plus général
mais cela laisse la porte ouverte à des dérivations TAG qui s’éloignent de l’interprétation sémantique :
\(\mathrm{Boire}(\mathrm{Pierre}, \mathrm{vin}(\mathrm{un\mbox{-}verre}))\) peut produire plutôt des lectures
\(\mathrm{Boire}(\mathrm{Pierre}, \mathrm{un\mbox{-}verre}(\mathrm{vin}))\).
Aussi, un arbre pour \og un verre de\fg{} semble plus restreindre aux contextes d’usage dans lesquels il s’agit
d’une mesure/quantité.

\medskip
Soit \(\alpha_{\text{vin}}\) :
\[
\begin{forest}
for tree={align=center, l sep=8mm, s sep=6mm, scale=0.92}
[N
  [vin]
]
\end{forest}
\]

\paragraph{Dérivation :}
\begin{itemize}
  \item Substituer \(\alpha_{\text{Pierre}}\) sur \(GN_{\text{agent}}\) de \(\alpha_{\text{boit}}\).
  \item Substituer \(\alpha_{\text{vin}}\) au \(N\) de \(\alpha_{\text{un-verre-de}}\).
  \item Substituer l’arbre résultant à \(GN_{\text{patient}}\) de \(\alpha_{\text{boit}}\).
\end{itemize}

\begin{center}
\begin{forest}
for tree={
  align=center,
  l sep=10mm,
  s sep=8mm,
  scale=0.92
}
[S
  [GN
    [Pierre]
  ]
  [GV
    [V [boit]]
    [GN
      [Det [un]]
      [N'
        [N [verre]]
        [GP
          [P [de]]
          [N [vin]]
        ]
      ]
    ]
  ]
]
\end{forest}

\end{center}

\end{document}
